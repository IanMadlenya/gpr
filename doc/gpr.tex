\documentclass[12pt]{article}

\usepackage{amsmath}
\usepackage{amsfonts}
\usepackage{amssymb}
\usepackage{amsbsy}
\usepackage{accents}

\DeclareMathAlphabet{\mathsfsl}{OT1}{cmss}{m}{sl}
\newcommand{\mat}[1]{\mbox{$\mathsfsl{#1}$}}
\renewcommand{\vec}[1]{\mbox{\boldmath$#1$}}
\newcommand{\dmat}[1]{\mat{\dot{{#1}}}}
\newcommand{\tr}{\mathrm{tr}}

\newcommand{\km}{\mat{K_{M}}}
\newcommand{\dkm}{\mat{\dot{K}_{M}}}
\newcommand{\dkn}{\mat{\dot{K}_{N}}}
\newcommand{\kmn}{\mat{K_{MN}}}
\newcommand{\dkmn}{\mat{\dot{K}_{MN}}}
\newcommand{\dukmn}{\myu{\mat{\dot{K}}}_{\mathsfsl{MN}}}
\newcommand{\lam}{\mat{\Lambda}}
\newcommand{\inv}[1]{#1^{-1}}
\newcommand{\ichol}[1]{#1^{-\frac{1}{2}}}
\newcommand{\mydef}{\stackrel{\mathrm{def}}{=}}
\newcommand{\myu}[1]{\underaccent{\bar}{#1}}
\newcommand{\myuu}[1]{\myu{\myu{#1}}}
\newcommand{\dlam}{\dmat{\Lambda}}
\newcommand{\dlamuu}{\myuu{\dmat{\Lambda}}}
\newcommand{\diagv}{\mathrm{diag_v}}
\newcommand{\diagm}{\mathrm{diag_m}}

\begin{document}

\section{FIC-Likelihood Derivatives}

These are the derivatives of the FIC-likelihood as used in the
OCaml-implementation.  The implementation factorizes the computations
in exactly that way to minimize computation time.  Numerical stability
may need to be improved.

diagm is the matrix consisting of only the diagonal
diagv is the diagonal matrix

TODO: explain trace product

precomputations

\begin{eqnarray*}
\mathfrak{A} & = & \mat{B}^{-1} - \mat{K_M}^{-1} \\
\mathfrak{B} & = & \ichol{\mat{B}}\kmn \\
\mathfrak{C} & = & \mat{B}^{-\frac{\top}{2}}\mathfrak{B} = \inv{\mat{B}}\kmn \\
\mathfrak{D} & = & \inv{\km}\kmn \\
\mathfrak{a} & = & \diagv(\mat{I} - \myu{\mathfrak{B}}^\top\myu{\mathfrak{B}}) \\
\mathfrak{b} & = & \mathfrak{C} \myuu{y} \\
\mathfrak{c} & = & \kmn^\top\mathfrak{b} - \vec{y} \\
\mathfrak{d} & = & \inv{\lam}\mathfrak{c} \\
\end{eqnarray*}

\begin{eqnarray*}
\dlam & = & \diagm(\dkn - 2\,\dkmn^\top\mathfrak{D} + \mathfrak{D}^\top\dkm\mathfrak{D}) \\
2\,\dot{L}_1 & = & \tr(\mathfrak{A}\dkm) + 2\,\tr({\mathfrak{C}^\top\dukmn}) + \mathfrak{a}^\top\diagv(\dlamuu) \\
2\,\dot{L}_2 & = & \mathfrak{b}^\top(\dkm\mathfrak{b} + 2\,\dkmn\mathfrak{d})
  + (\mathfrak{c}:\vec{y} - (\mathfrak{c} + \vec{y}) : \mathfrak{c})^\top\diagv(\dlam) \\
\end{eqnarray*}

\end{document}
