\documentclass[10pt]{report}

% PACKAGES
\usepackage[usenames]{color}

\newcommand{\mail}{\mailto{markus.mottl@gmail.com}}
\newcommand{\athome}[2]{\ahref{http://www.ocaml.info/#1}{#2}}
\newcommand{\www}{\athome{}{Markus Mottl}}

% INCLUDE HEVEA SUPPORT
\usepackage{hevea}

%BEGIN LATEX
\usepackage{natbib}
%END LATEX

\usepackage{amsmath}
\usepackage{amsfonts}
\usepackage{amssymb}
\usepackage{amsbsy}
\usepackage{accents}

% HTML FOOTER
\htmlfoot{
  \rule{\linewidth}{1mm}
  Copyright \quad \copyright \quad 2009-
  \quad \www \quad \langle\mail\rangle
}

% HYPHENATION

\hyphenation{he-te-ro-ske-da-stic}
\hyphenation{ana-ly-ti-cally}
\hyphenation{know-ledge}

\DeclareMathAlphabet{\mathsfsl}{OT1}{cmss}{m}{sl}

\newcommand{\red}{\textcolor{red}}
\newcommand{\blue}{\textcolor{blue}}

\newcommand{\dif}{\mathrm{d}}

\newcommand{\myu}[1]{\underaccent{\bar}{#1}}

\newcommand{\onehalf}{\tfrac{1}{2}}

\newcommand{\mat}[1]{\mbox{$\mathsfsl{#1}$}}
\newcommand{\myvec}[1]{\mbox{\boldmath$#1$}}

\newcommand{\diagv}[1]{\mathrm{diag_v}(#1)}
\newcommand{\diagm}[1]{\mathrm{diag_m}(#1)}
\newcommand{\trace}[1]{\mathrm{tr}(#1)}
\newcommand{\transv}[1]{\myvec{#1}^\top}
\newcommand{\transm}[1]{\mat{#1}^\top}

\newcommand{\imat}[1]{\mat{#1^{-1}}}
\newcommand{\itransm}[1]{\mat{#1^{-\top}}}
\newcommand{\chol}[1]{\mat{#1^{\onehalf}}}
\newcommand{\cholt}[1]{\mat{#1^{\tfrac{\top}{2}}}}
\newcommand{\ichol}[1]{\mat{#1^{-\onehalf}}}
\newcommand{\icholt}[1]{\mat{#1^{-\tfrac{\top}{2}}}}

\newcommand{\Km}{\mat{K_M}}
\newcommand{\iKm}{\imat{K_M}}
\newcommand{\dKm}{\mat{\dot{K}_M}}
\newcommand{\dKn}{\mat{\dot{K}_N}}
\newcommand{\Knm}{\mat{K_{NM}}}
\newcommand{\Kmn}{\transm{K_{NM}}}
\newcommand{\uKnm}{\myu{\mat{K}}_{\mathsfsl{NM}}}
\newcommand{\uuKnm}{\myu{\myu{\mat{K}}}_{\mathsfsl{NM}}}
\newcommand{\dKnm}{\mat{\dot{K}_{NM}}}
\newcommand{\uKmn}{\myu{\mat{K}}_{\mathsfsl{NM}}^\top}

\newcommand{\dl}{\dot{l}}

\newcommand{\vecu}{\myvec{u}}
\newcommand{\vecr}{\myvec{r}}
\newcommand{\vecs}{\myvec{s}}
\newcommand{\vect}{\myvec{t}}
\newcommand{\vecw}{\myvec{w}}
\newcommand{\vecv}{\myvec{v}}
\newcommand{\vecvx}{\myvec{v}_1}
\newcommand{\vecvy}{\myvec{v}_2}
\newcommand{\vecy}{\myvec{y}}
\newcommand{\uvecy}{\myu{\vecy}}

\newcommand{\vecsdh}{\onehalf\myvec{\dot{s}}}
\newcommand{\vecis}{\myvec{s}^{-1}}
\newcommand{\veciss}{\myvec{s}^{-\onehalf}}

\newcommand{\matB}{\mat{B}}
\newcommand{\matI}{\mat{I}}
\newcommand{\matQ}{\mat{Q}}
\newcommand{\matQn}{\mat{\widetilde{Q}}}
\newcommand{\tmatQn}{\transm{\widetilde{Q}}}
\newcommand{\matR}{\mat{R}}
\newcommand{\matS}{\mat{S}}
\newcommand{\matT}{\mat{T}}
\newcommand{\matU}{\mat{U}}
\newcommand{\matUx}{\mat{U}_1}
\newcommand{\matUy}{\mat{U}_2}
\newcommand{\matV}{\mat{V}}
\newcommand{\matW}{\mat{W}}
\newcommand{\matWx}{\mat{W}_1}
\newcommand{\matX}{\mat{X}}
\newcommand{\matXx}{\mat{X}_1}
\newcommand{\matWy}{\mat{W}_2}
\newcommand{\matXy}{\mat{X}_2}

\newcommand{\Lam}{\mat{\Lambda}}
\newcommand{\Lamss}{\mat{\Lambda}_{\sigma^2}}
\newcommand{\Lamssi}{\imat{\Lambda_{\sigma^2}}}

% TITLE

\begin{titlepage}

\title{Gaussian Process Regression with OCaml\\Version 0.9}

\author{Markus Mottl\footnote{\mail}}

\date{\today}

\end{titlepage}

% DOCUMENT
\begin{document}

\maketitle

\begin{abstract}

This manual documents the implementation and use of the OCaml GPR library for
Gaussian Process Regression with OCaml.

\end{abstract}

\tableofcontents

\chapter{Overview}

The OCaml GPR library features implementations of many of the latest
developments in the currently heavily researched machine learning area of
Gaussian process regression.

\section{Background}

Gaussian processes define probability distributions over functions as prior
knowledge.  Bayesian inference can then be used to compute posterior
distributions over these functions given data, e.g.\ to solve regression
problems\footnote{Gaussian processes can also be used for classification
purposes.  This is by itself a large research area, which is not covered by this
library.}.  As more data becomes available, the Gaussian process framework
learns an ever more accurate distribution of functions that generate the data.\\

Due to their mathematically elegant nature, Gaussian processes allow for
analytically tractable calculation of the posterior mean and covariance
functions.  Though it is easy to formulate the required equations, GPs come at a
usually intractably high computational price for large problems.  Efficient
approximation methods have been developed in the recent past to address this
shortcoming, and this library makes heavy use of them.\\

Gaussian processes are true generalizations of e.g.\ linear regression, ARMA
processes, single-layer neural networks with an infinite number of hidden units,
and many other more widely known modeling techniques.  GPs are closely related
to support vector- (SVM) and other kernel machines, but have features that may
make them a more suitable choice in many situations.  For example they offer
predictive variances, Bayesian model selection, sampling from the posterior
distribution, etc.\\

It would go beyond the scope of this library documentation to provide for a
detailed treatment of Gaussian processes.  Hence, readers unfamiliar with this
approach may want to consult online resources, of which there are plenty.  This
section presents an overview of recommended materials.

\subsection{Video tutorials}

Video tutorials are probably best suited for quickly developing an intuition and
basic formal background of Gaussian processes and perspectives for their
practical use.

\begin{itemize}

\item \emph{\footahref{http://videolectures.net/gpip06\_mackay\_gpb}{Gaussian
Process Basics}}: David MacKay's lecture given at the \emph{Gaussian Processes
in Practice Workshop} in 2006.  This one hour video tutorial uses numerous
graphical examples and animations to aid understanding of the basic principles
behind inference techniques based on Gaussian processes.

\item
\emph{\footahref{http://videolectures.net/epsrcws08\_rasmussen\_lgp}{Learning
with Gaussian Processes}}: a slightly longer, two hour video tutorial series
presented by Carl Edward Rasmussen at the Sheffield EPSRC Winter School 2008,
which goes into somewhat more detail.

\item
\emph{\footahref{http://videolectures.net/mlss07\_rasmussen\_bigp}{Bayesian
Inference and Gaussian Processes}}: readers interested in a fairly thorough,
from the ground up treatment of Bayesian inference techniques using Gaussian
processes may want to watch this five hour video tutorial series presented by
Carl Edward Rasmussen at the MLSS 2007 in T\"ubingen.

\end{itemize}

\subsection{Books and papers}

The following texts are intended for people who need a more formal treatment and
theory.  This is especially recommended if you want to be able to implement
Gaussian processes and their approximations efficiently.

\begin{itemize}

\item
\emph{\footahref{http://www.gatsby.ucl.ac.uk/\home{snelson}/thesis.pdf}{Flexible
and efficient Gaussian process models for machine learning}}: Edward Lloyd
Snelson's PhD thesis (\cite{SnelsonThesis}) offers a particularly readable
treatment of modern inference and approximation techniques that avoids heavy
formalism in favor of intuitive notation and clearly presented high-level
concepts without sacrificing detail needed for implementation.  This library
owes a lot to his work.

\item \emph{\footahref{http://www.gaussianprocess.org/gpml}{Gaussian Processes
for Machine Learning}}: many researchers in this area would call this book,
which was written by Carl Edward Rasmussen and Christopher K.\ I.\  Williams,
the ``bible of Gaussian processes''.  It presents a rigorous treatment of the
underlying theory for both regression and classification problems, and more
general aspects like properties of covariance functions, etc.  The authors have
kindly made the full text and Matlab sources available online.  Their
\footahref{http://www.gaussianprocess.org}{Gaussian process website} also lists
a great wealth of other resources valuable for both researchers and
practitioners.

\end{itemize}

References to research about specific techniques used in the OCaml GPR library
are provided in the bibliography.

\section{Features of OCaml GPR}

Among other things the OCaml GPR library currently offers:

\begin{itemize}

\item Sparse Gaussian processes using the FI(T)C\footnote{\emph{Fully
Independent (Training) Conditional}} approximations for computationally
tractable learning (see \cite{conf/nips/2005}, \cite{SnelsonThesis}).  Unlike
some other approximations that lead to degeneracy, FI(T)C maintains sane
posterior variances.

\item Safe and convenient API for computing posterior means, variances,
covariances, log evidence, for sampling from the posterior distribution,
calculating statistics of the quality of fit, etc.

\item Optimization of hyper parameters via evidence maximization\footnote{Also
known as type II maximum likelihood.}, including optimization of inducing inputs
(SPGP algorithm\footnote{This library exploits sparse matrix operations to
achieve optimum big-O complexity when learning inducing inputs with the SPGP
algorithm, but also for multiscales and other hyper parameters that imply sparse
derivative matrices.}).

\item Supervised dimensionality reduction, and improved predictions under
heteroskedastic noise conditions (see \cite{conf/uai/SnelsonG06},
\cite{SnelsonThesis}).

\item Sparse multiscale Gaussian process regression (see
\cite{conf/icml/WalderKS08}).

\item Variational improvements to the approximate posterior distribution (see
\cite{Titsias2009}).

\item Numerically stable GP calculations using QR-factorization to avoid the
more commonly used and numerically unstable solution of normal equations via
Cholesky factorization (see \cite{Foster2009}).

\item Consistent use of BLAS/LAPACK throughout the library for optimum
performance.

\item Functors for plugging arbitrary covariance functions (= kernels) into the
framework.  There is no constraint on the type of covariance functions, i.e.\
also string inputs, graph inputs, etc., could potentially be used with ease
given suitable covariance functions\footnote{The library is currently only
distributed with covariance functions that operate on multivariate numerical
inputs.  Interested readers may feel free to contribute others.}.

\item Rigorous test suite for checking both user-provided derivatives of
covariance functions, which are usually quite hard to implement correctly, and
self-test code to verify derivatives of log likelihood functions using finite
differences.

\end{itemize}

\chapter{API documentation}

\chapter{Example application}

\chapter{Internals}

\appendix

\chapter{FIC computations}

This section consists of equations used for computing the FI(T)C predictive
distribution, and the log likelihood and its derivatives in the OCaml GPR
library.  The implementation factorizes the computations in this way for several
reasons: to minimize computation time and memory usage, and to improve numerical
stability by e.g.\ using QR factorization to avoid normal equations, and by
avoiding inverses whenever possible without great loss of
efficiency\footnote{Unfortunately, inverses and symmetric rank-k operations are
unavoidable in some cases to preserve optimum big-O complexity.}.  It otherwise
aims for ease of implementation, e.g.\ combining derivative terms to simplify
dealing with sparse matrices.\\

The presentation and notation here is somewhat similar to \cite{SnelsonThesis}.
Thus, interested readers are encouraged to first read his work, especially the
derivations in the appendix.  Our presentation deviates in minor ways, but
should hopefully still be fairly easy to compare.  The log likelihood
derivatives have been heavily restructured though.  The mathematical derivation
of this restructuring would be extremely tedious, hence only the final result is
presented.\\

Here are a few definitions:

\begin{itemize}

\item $\mathrm{diag_m}$ is the function that returns the matrix consisting of
only the diagonal of a given matrix.  $\mathrm{diag_v}$ returns the diagonal as
a vector.

\item $\otimes$ represents element-wise multiplication of vectors.  A vector
raised to a power means element-wise application of that power.

\item Parts in \red{red} represent terms used for Michalis K.\ Titsias'
variational improvement to the posterior marginal likelihood (see
\cite{Titsias2009}).

\item Parts in \blue{blue} provide for an alternative, more compact, direct and
hence more efficient way of computing some result if the required terms are
already available.

\end{itemize}

\begin{eqnarray*}
\matV & = & \Knm\icholt{K_M} \\
\mat{\widetilde{K}_N} & = & \matV \transm{V} \\
\Lam & = & \diagm{\mat{K_N} - \mat{\widetilde{K}_N}} \\
\Lamss & = & \Lam + \sigma^2 \matI \\
\\
\vecr & = & \diagv{\Lam} \\
\vecs & = & \diagv{\Lamss} \\
\\
\uKnm & = & \ichol{\Lamss} \Knm \\
\matQ \matR & = & {\uKnm\choose\cholt{K_M}} \hspace{5mm}
\textrm{(QR-factorization of $\uKnm$ stacked on $\cholt{K_M}$)} \\
\\
\matB & = & \Km + \uKmn\uKnm = \transm{R}\transm{Q} \mat{Q} \matR = \transm{R}\matR \\
\matQn & = & {\lfloor \matQ \rfloor\footnotemark}_{N} \longrightarrow \uKnm = \matQn \matR \\
\matS & = & \ichol{\Lamss}\matQn\itransm{R} \\
\\
l_1 & = & -\onehalf (\log|\matB| - \log|\Km| + \log|\Lamss| + N \log 2\pi) \red{+ -\onehalf\vecis \cdot \vecr} \\
\\
\uvecy & = & \veciss \otimes \vecy \\
\vect & = & \imat{R} \tmatQn \uvecy \\
\vecu & = & \uvecy - \matQn \tmatQn \uvecy \\
\\
l_2 & = & \blue{-\onehalf \vecu\cdot\uvecy} = -\onehalf(\|\uvecy\|^2 - \|\tmatQn \uvecy\|^2) \\
l & = & l_1 + l_2 \\
\\
\matT & = & \imat{K_M} - \imat{B} \\
\\
\mu_* & = & \mat{K_{*M}} \vect \\
\sigma^2_* & = & K_* - \mat{K_{*M}}\matT\transm{K_{*M}} + \sigma^2 \matI \\
\end{eqnarray*}
\footnotetext{Take first $N$ rows.}

\begin{eqnarray*}
\matU & = & \mat{V} \ichol{K_M} \\
\\
\vecvx & = & \vecis \otimes (\vec{1} \red{\, + \, \vec{1} - \vecis \otimes \vecr} - \diagv{\matQn\tmatQn}) \\
\matUx & = & \diagm{\vecvx^{\onehalf}} \matU \\
\matWx & = & \matT - \matUx^\top\matUx \\
\matXx & = & \matS - \diagm{\vecvx}\matU \\
\dl_1 & = & -\onehalf(\vecvx \cdot \diagv{\dKn} - \trace{\transm{W}_1\dKm}) - \trace{\transm{X}_1\dKnm} \\
\\
\vecw & = & \veciss \otimes \vecu \\
\vecvy & = & \vecw \otimes \vecw \\
\matUy & = & \diagm{\vecw} \matU \\
\matWy & = & \vect \vect^\top - \matUy^\top\matUy \\
\matXy & = & \vecw\vect^\top - \diagm{\vecvy}\matU \\
\dl_2 & = & \onehalf(\vecvy \cdot \diagv{\dKn} - \trace{\transm{W}_2\dKm}) + \trace{\transm{X}_2\dKnm} \\
\\
\dl & = & \dl_1 + \dl_2 \\
\\
\tfrac{\partial l_1}{\partial\sigma^2} & = & -\onehalf(\mathrm{sum}(\vecvx) \red{\, - \, \mathrm{sum}(\vecis)}) \\
\tfrac{\partial l_2}{\partial\sigma^2} & = & \onehalf\mathrm{sum}(\vecvy) \\
\tfrac{\partial l}{\partial\sigma^2} & = & \tfrac{\partial l_1}{\partial\sigma^2} + \tfrac{\partial l_2}{\partial\sigma^2}
\end{eqnarray*}
\blue{
\begin{eqnarray*}
\vecv & = & \vecvx - \vecvy  \\
\matW & = & \matWx - \matWy = \matT - \vect \transv{t} - \matUx^\top\matUx + \matUy^\top\matUy \\
\matX & = & \matXx - \matXy = \matS - \vecw \transv{t} - \diagm{\vecv}\matU \\
\dl & = & -\onehalf(\vecv \cdot \diagv{\dKn} - \trace{\transm{W}\dKm}) - \trace{\transm{X}\dKnm} \\
\\
\tfrac{\partial l}{\partial\sigma^2} & = & -\onehalf(\mathrm{sum}(\vecv) \red{\, - \, \mathrm{sum}(\vecis)}) \\
\end{eqnarray*}
}

\chapter{Notes/reminders for future work}

Initialize inducing inputs with partial Cholesky factorization
(better with discrete values)

Log-derivatives:

\begin{itemize}
\item $\tfrac{\partial f}{\partial \log(x)} = \tfrac{\partial f}{\partial x} x$
\end{itemize}

\section{Nonlinear clustering:}

\begin{itemize}
\item $k(x, y) = \langle \phi(x) | \phi(y) \rangle$
\item $\|\phi(x) - \phi(y)\|^2 = k(x,x)-2k(x,y)+k(y,y)$
\item find one inducing point
\item choose point x farthest away wrt.\ k
\item choose antipodal point y to x wrt.\ k
\item determine for all points to which of x or y they are closer
\item create two clusters
\item recurse
\item when suitable granularity reached, use PI(T)C
\end{itemize}

% BIBLIOGRAPHY
\bibliographystyle{alpha}
\bibliography{gpr}

\end{document}
