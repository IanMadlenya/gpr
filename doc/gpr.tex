\documentclass[10pt]{article}

\usepackage[usenames]{color}

\usepackage{amsmath}
\usepackage{amsfonts}
\usepackage{amssymb}
\usepackage{amsbsy}
\usepackage{accents}

\DeclareMathAlphabet{\mathsfsl}{OT1}{cmss}{m}{sl}

\newcommand{\red}{\textcolor{red}}
\newcommand{\blue}{\textcolor{blue}}

\newcommand{\dif}{\mathrm{d}}

\newcommand{\myu}[1]{\underaccent{\bar}{#1}}

\newcommand{\onehalf}{\tfrac{1}{2}}

\newcommand{\mat}[1]{\mbox{$\mathsfsl{#1}$}}
\newcommand{\myvec}[1]{\mbox{\boldmath$#1$}}

\newcommand{\diagv}[1]{\mathrm{diag_v}(#1)}
\newcommand{\diagm}[1]{\mathrm{diag_m}(#1)}
\newcommand{\trace}[1]{\mathrm{tr}(#1)}
\newcommand{\transv}[1]{\myvec{#1}^\top}
\newcommand{\transm}[1]{\mat{#1}^\top}

\newcommand{\imat}[1]{\mat{#1^{-1}}}
\newcommand{\itransm}[1]{\mat{#1^{-\top}}}
\newcommand{\chol}[1]{\mat{#1^{\onehalf}}}
\newcommand{\cholt}[1]{\mat{#1^{\tfrac{\top}{2}}}}
\newcommand{\ichol}[1]{\mat{#1^{-\onehalf}}}
\newcommand{\icholt}[1]{\mat{#1^{-\tfrac{\top}{2}}}}

\newcommand{\Km}{\mat{K_M}}
\newcommand{\iKm}{\imat{K_M}}
\newcommand{\dKm}{\mat{\dot{K}_M}}
\newcommand{\dKn}{\mat{\dot{K}_N}}
\newcommand{\Knm}{\mat{K_{NM}}}
\newcommand{\Kmn}{\transm{K_{NM}}}
\newcommand{\uKnm}{\myu{\mat{K}}_{\mathsfsl{NM}}}
\newcommand{\uuKnm}{\myu{\myu{\mat{K}}}_{\mathsfsl{NM}}}
\newcommand{\dKnm}{\mat{\dot{K}_{NM}}}
\newcommand{\uKmn}{\myu{\mat{K}}_{\mathsfsl{NM}}^\top}

\newcommand{\dl}{\dot{l}}

\newcommand{\vecu}{\myvec{u}}
\newcommand{\vecr}{\myvec{r}}
\newcommand{\vecs}{\myvec{s}}
\newcommand{\vect}{\myvec{t}}
\newcommand{\vecw}{\myvec{w}}
\newcommand{\vecv}{\myvec{v}}
\newcommand{\vecvx}{\myvec{v}_1}
\newcommand{\vecvy}{\myvec{v}_2}
\newcommand{\vecy}{\myvec{y}}
\newcommand{\uvecy}{\myu{\vecy}}

\newcommand{\vecsdh}{\onehalf\myvec{\dot{s}}}
\newcommand{\vecis}{\myvec{s}^{-1}}
\newcommand{\veciss}{\myvec{s}^{-\onehalf}}

\newcommand{\matB}{\mat{B}}
\newcommand{\matI}{\mat{I}}
\newcommand{\matQ}{\mat{Q}}
\newcommand{\matQn}{\mat{\widetilde{Q}}}
\newcommand{\tmatQn}{\transm{\widetilde{Q}}}
\newcommand{\matR}{\mat{R}}
\newcommand{\matS}{\mat{S}}
\newcommand{\matT}{\mat{T}}
\newcommand{\matU}{\mat{U}}
\newcommand{\matUx}{\mat{U}_1}
\newcommand{\matUy}{\mat{U}_2}
\newcommand{\matV}{\mat{V}}
\newcommand{\matW}{\mat{W}}
\newcommand{\matWx}{\mat{W}_1}
\newcommand{\matX}{\mat{X}}
\newcommand{\matXx}{\mat{X}_1}
\newcommand{\matWy}{\mat{W}_2}
\newcommand{\matXy}{\mat{X}_2}

\newcommand{\Lam}{\mat{\Lambda}}
\newcommand{\Lamss}{\mat{\Lambda}_{\sigma^2}}
\newcommand{\Lamssi}{\imat{\Lambda_{\sigma^2}}}

\begin{document}

\section{FIC computations}

These are the equations used for computing the FIC predictive
distribution and FIC marginal likelihood and its derivatives in the
OCaml-implementation.  The implementation factorizes the computations
in this way for several reasons: to minimize computation time and
memory usage, and to improve numerical stability by e.g.\ avoiding
inverses and by using QR factorization to avoid normal equations
whenever possible without great loss of efficiency.  It otherwise
aims for ease of implementation, e.g.\ combining derivative terms
to simplify dealing with sparse matrices.\\

Here are a few symbology conventions:

\begin{itemize}
\item $\mathrm{diag_m}$ is the matrix consisting of only the diagonal
\item Parts in \red{red} represent terms used for Michalis Titsias'
variational approximation of the posterior marginal likelihood.
\item Parts in \blue{blue} provide for an alternative, more compact,
direct and hence more efficient way of computing some result if the
required parameters are already available.
\end{itemize}

\begin{eqnarray*}
\matV & = & \Knm\icholt{K_M} \\
\mat{\widetilde{K}_N} & = & \matV \transm{V} \\
\Lam & = & \diagm{\mat{K_N} - \mat{\widetilde{K}_N}} \\
\Lamss & = & \Lam + \sigma^2 \matI \\
\\
\vecr & = & \diagv{\Lam} \\
\vecs & = & \diagv{\Lamss} \\
\\
\uKnm & = & \ichol{\Lamss} \Knm \\
\matQ \matR & = & {\uKnm\choose\cholt{K_M}} \hspace{1cm} \textrm{(QR-factorization)} \\
\\
\matB & = & \Km + \uKmn\uKnm = \transm{R}\transm{Q} \mat{Q} \matR = \transm{R}\matR \\
\matQn & = & {\lfloor \matQ \rfloor}_{N}\footnotemark \longrightarrow \uKnm = \matQn \matR \\
\matS & = & \ichol{\Lamss}\matQn\itransm{R} \\
\\
l_1 & = & -\onehalf (\log|\matB| - \log|\Km| + \log|\Lamss| + N \log 2\pi) \red{+ -\onehalf\vecis \cdot \vecr} \\
\\
\uvecy & = & \veciss \otimes \vecy \\
\vect & = & \imat{R} \tmatQn \uvecy \\
\vecu & = & \uvecy - \matQn \tmatQn \uvecy \\
\\
l_2 & = & \blue{-\onehalf \vecu\cdot\uvecy} = -\onehalf(\|\uvecy\|^2 - \|\tmatQn \uvecy\|^2) \\
l & = & l_1 + l_2 \\
\\
\matT & = & \imat{K_M} - \imat{B} \\
\\
\mu_* & = & \mat{K_{*M}} \vect \\
\sigma^2_* & = & K_* - \mat{K_{*M}}\matT\transm{K_{*M}} + \sigma^2 \matI \\
\end{eqnarray*}
\footnotetext{Take first $N$ rows.}

\begin{eqnarray*}
\matU & = & \mat{V} \ichol{K_M} \\
\\
\vecvx & = & \vecis \otimes (\vec{1} \red{\, + \, \vec{1} - \vecis \otimes \vecr} - \diagv{\matQn\tmatQn}) \\
\matUx & = & \diagm{\vecvx^{\onehalf}} \matU \\
\matWx & = & \matT - \matUx^\top\matUx \\
\matXx & = & \matS - \diagm{\vecvx}\matU \\
\dl_1 & = & -\onehalf(\vecvx \cdot \diagv{\dKn} - \trace{\transm{W}_1\dKm}) - \trace{\transm{X}_1\dKnm} \\
\\
\vecw & = & \veciss \otimes \vecu \\
\vecvy & = & \vecw \otimes \vecw \\
\matUy & = & \diagm{\vecw} \matU \\
\matWy & = & \vect \vect^\top - \matUy^\top\matUy \\
\matXy & = & \vecw\vect^\top - \diagm{\vecvy}\matU \\
\dl_2 & = & \onehalf(\vecvy \cdot \diagv{\dKn} - \trace{\transm{W}_2\dKm}) + \trace{\transm{X}_2\dKnm} \\
\\
\dl & = & \dl_1 + \dl_2 \\
\\
\tfrac{\partial l_1}{\partial\sigma^2} & = & -\onehalf(\mathrm{sum}(\vecvx) \red{\, - \, \mathrm{sum}(\vecis)}) \\
\tfrac{\partial l_2}{\partial\sigma^2} & = & \onehalf\mathrm{sum}(\vecvy) \\
\tfrac{\partial l}{\partial\sigma^2} & = & \tfrac{\partial l_1}{\partial\sigma^2} + \tfrac{\partial l_2}{\partial\sigma^2}
\end{eqnarray*}
\blue{
\begin{eqnarray*}
\vecv & = & \vecvx - \vecvy  \\
\matW & = & \matWx - \matWy = \matT - \vect \transv{t} - \matUx^\top\matUx + \matUy^\top\matUy \\
\matX & = & \matXx - \matXy = \matS - \vecw \transv{t} - \diagm{\vecv}\matU \\
\dl & = & -\onehalf(\vecv \cdot \diagv{\dKn} - \trace{\transm{W}\dKm}) - \trace{\transm{X}\dKnm} \\
\\
\tfrac{\partial l}{\partial\sigma^2} & = & -\onehalf(\mathrm{sum}(\vecv) \red{\, - \, \mathrm{sum}(\vecis)}) \\
\end{eqnarray*}
}

\newpage

\section{Notes/reminders for future work}

Log-derivatives:

\begin{itemize}
\item $\tfrac{\partial f}{\partial \log(x)} = \tfrac{\partial f}{\partial x} x$
\end{itemize}

\subsection{Nonlinear clustering:}

\begin{itemize}
\item $k(x, y) = \langle \phi(x) | \phi(y) \rangle$
\item $\|\phi(x) - \phi(y)\|^2 = k(x,x)-2k(x,y)+k(y,y)$
\item find one inducing point
\item choose point x farthest away wrt.\ k
\item choose antipodal point y to x wrt.\ k
\item determine for all points to which of x or y they are closer
\item create two clusters
\item recurse
\item when suitable granularity reached, use PI(T)C
\end{itemize}

\end{document}
