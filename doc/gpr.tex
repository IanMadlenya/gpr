\documentclass[12pt]{article}

\usepackage{amsmath}
\usepackage{amsfonts}
\usepackage{amssymb}
\usepackage{amsbsy}
\usepackage{accents}

\DeclareMathAlphabet{\mathsfsl}{OT1}{cmss}{m}{sl}

\newcommand{\myu}[1]{\underaccent{\bar}{#1}}

\newcommand{\mat}[1]{\mbox{$\mathsfsl{#1}$}}
\renewcommand{\vec}[1]{\mbox{\boldmath$#1$}}

\newcommand{\diagv}[1]{\mathrm{diag_v}(#1)}
\newcommand{\diagm}[1]{\mathrm{diag_m}(#1)}
\newcommand{\trace}[1]{\mathrm{tr}(#1)}
\newcommand{\transv}[1]{\vec{#1^\top}}
\newcommand{\transm}[1]{\mat{#1^\top}}

\newcommand{\imat}[1]{\mat{#1^{-1}}}
\newcommand{\ichol}[1]{\mat{#1^{-\frac{1}{2}}}}
\newcommand{\icholt}[1]{\mat{#1^{-\frac{\top}{2}}}}

\newcommand{\Km}{\mat{K_M}}
\newcommand{\iKm}{\imat{K_M}}
\newcommand{\dKm}{\mat{\dot{K}_M}}
\newcommand{\dkn}{\mat{\dot{K}_N}}
\newcommand{\Kmn}{\mat{K_{MN}}}
\newcommand{\Knm}{\transm{K_{MN}}}
\newcommand{\dKmn}{\mat{\dot{K}_{MN}}}
\newcommand{\uuKmn}{\myu{\myu{\mat{K}}}_{\mathsfsl{MN}}}

\newcommand{\vecs}{\vec{s}}
\newcommand{\vect}{\vec{t}}
\newcommand{\vecu}{\vec{u}}
\newcommand{\vecv}{\vec{v}}
\newcommand{\vecw}{\vec{w}}
\newcommand{\vecy}{\vec{y}}

\newcommand{\vecsd}{\vec{\dot{s}}}
\newcommand{\vecsi}{\vec{\tilde{s}}}

\newcommand{\matS}{\mat{S}}
\newcommand{\matT}{\mat{T}}
\newcommand{\matU}{\mat{U}}
\newcommand{\matV}{\mat{V}}

\newcommand{\Lamss}{\mat{\Lambda_{\sigma^2}}}
\newcommand{\Lamssi}{\mat{\Lambda_{\sigma^2}^{-1}}}

\begin{document}

\section{FIC-Likelihood Derivatives}

These are the derivatives of the FIC-likelihood as used in the
OCaml-implementation.  The implementation factorizes the computations
in exactly that way to minimize computation time.  Numerical stability
may need to be improved.

diagm is the matrix consisting of only the diagonal
dprod is diagonal of the matrix product

definitions / precomputations

general derivative, not noise here!

\begin{eqnarray*}
\Lamss & = & \diagm{\mat{K_N}} - \diagm{\mat{Q_N}} + \sigma^2\mat{I} \\
\uuKmn & = & \Kmn\Lamssi \\
\matS & = & \imat{B} - \imat{K_M} \\
\matT & = & \ichol{B}\uuKmn \\
\matU & = & \imat{B}\uuKmn = \icholt{B}\matT \\
\matV & = & \iKm\Kmn \\
\vecs & = & \diagv{\Lamss} \\
\vecsi & = & \diagv{\Lamssi} \\
\vecsd & = & \diagv{\dkn + \transm{V}(\dKm V - 2\,\dKmn)} \\
\vect & = & \frac{1}{2}(\diagv{\transm{T}\matT} - \vec{\mathbf{\vecsi}}) \\
\vecu & = & \matU \vecy \\
\vecv & = & \Knm \vecu - \vecy \\
\vecw & = & \vecsi\otimes \vecv \\
\dot{L}_1 & = & \frac{1}{2}\trace{\matS\dKm} - \trace{\transm{U}\dKmn} + \transv{t}\vecsd \\
2\,\dot{L}_2 & = & \vecu\cdot(\dKm \vecu + 2\,\dKmn \vecw) - (\vecv\otimes \vecv)\cdot\vecsd \\
\end{eqnarray*}

\end{document}
